\documentclass[a4paper, 11pt]{article}
\usepackage{amsmath}
\usepackage{amssymb}
\usepackage{amsthm}

\usepackage{witharrows}
\usepackage[utf8]{inputenc}
\usepackage[slovene]{babel}

\usepackage{tikz}     % tole je za koordinatni sistem
\usepackage{verbatim} % tole sm vidu zraven, najbrž rabš
\usetikzlibrary{angles, arrows, quotes}

\title{Transformacije prostora}
\author{Jošt Smrtnik}

\begin{document}
    \maketitle
    \section{Opis točk v prostoru}
    Točke lahko opišemo kot krajevne vektorje v $N$ dimenzionalnem prostoru.
    \begin{equation}
        \text{točka $P$: }\vec{P} = \left[
        \begin{matrix}
            x_1\\x_2\\x_3\\ \vdots \\ x_N
        \end{matrix}
        \right]
    \end{equation}
    Položaj opazovalca je tudi točka($O$), ki jo bom označeval z vektorjem $\vec{O}$. S pomočjo tega, lahko izračunamo relativni vektor $P'$, ki pomeni vektor med točko $O$ in točko $P$.
    \begin{equation}
        P' = \vec{P} - \vec{O}
    \end{equation}
    Od tu naprej bo računanje potekalo večinoma samo še s točkami $P'$.
    \section{Obračanje - rotacija}
    Obračanje lahko izvajamo s pomočjo matrik. V treh dimezijah pravimo, da vrtimo okrog neke osi. V več dimenzionalnem svetu, to ni več tako. Za primer, v 4 dimenzionalnem svetu potekajo rotacije okrog ravnine. Zaradi tega bom govoril o vrtenju vzporedno z ravnino, saj je to dovolj preprosto in deluje v poljubnem številu dimenzij večjem od 2. Pri takšni rotaciji, lahko govorimo celo o vrtenju od ene proti drugi osi. To pa omogoča predstavljanje vrtenja v dveh dimenzijah, saj potrebujemo prikaz samo dveh spremenjenih koordinat točke.\\
    \begin{figure}
        \begin{tikzpicture}[
            scale=4,
            axis/.style={very thick, ->, >=stealth'},
            important line/.style={thick},
            dashed line/.style={dashed, thin},
            pile/.style={thick, ->, >=stealth', shorten <=2pt, shorten
                >=2pt},
            every node/.style={color=black}
            ]
            
            
            % axis
            \draw[thick,-stealth] (-1.4,0) -- (1.4,0) node(xline)[right] {$x$};
            \draw[thick,-stealth] (0,-1.4) -- (0,1.4) node(yline)[above] {$y$};
            % Lines
            
           \node[inner sep=0pt,minimum size=4pt] (nula) at (0,0) {};
           \node[inner sep=0pt,minimum size=4pt] (a) at (0,1) {};
           \node[inner sep=0pt,minimum size=4pt] (b) at (1,0) {};

           \draw[-stealth, line width=0.5mm, shorten <= -1mm] (nula) to node [left=7pt] {$\vec{j}$} (a);
           \draw[-stealth, line width=0.5mm, shorten <= -1mm] (nula) to node [below=7pt] {$\vec{i}$} (b);
           
           % kot
           \draw pic[draw,angle radius=2cm, "$\alpha$" shift={(6mm,1mm)}] {angle=b--nula--a};
           %\draw pic[draw,fill=blue!30,angle radius=0.7cm,"$\epsilon$" shift={(-3mm,5mm)}] {angle=b--a--c};
           
           \coordinate (skornula) at (-0.01, 0);
           \draw(skornula) -- ++(0.9cm,0) arc(0:{atan(1/2)}:0.9cm) node[midway,left] {$\varphi$} -- cycle;
           
           \coordinate (skornula) at (-0.01, 0);
           \draw(skornula) -- ++(0,0.9cm) arc(0:{atan(1/2)+90}:0.9cm) node[midway,left] {$\varphi$} -- cycle;

           
           \node[above,font=\large\bfseries] at (current bounding box.north) {Prikaz vrtenja od $x$ proti $y$ osi};
        \end{tikzpicture}
    \end{figure}
    
\end{document}