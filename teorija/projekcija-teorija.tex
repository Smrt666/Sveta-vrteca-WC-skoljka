\documentclass[a4paper, 12px]{article}
\usepackage{amsmath}
\usepackage{amssymb}
\usepackage{amsthm}

\usepackage{biblatex}
\addbibresource{viri-projekcija.bib}

\usepackage{witharrows}

\usepackage{tikz}
\usetikzlibrary{automata, positioning, angles, quotes, calc, patterns}

\usepackage[utf8]{inputenc}
\usepackage[slovene]{babel}

\title{Projekcija \\[1ex] \large Del teorije pri projektnem delu v drugem letniku}

\begin{document}
\maketitle

\section{3D projekcija}
    3D projekcija oziroma grafična projekcija je tehnika, ki se uporablja za prikaz 3 dimenzionalnih objektov na 
    2 dimenzionalno ravnino. Obstaja več različnih vrst projekcij. V grobem se delijo na paralelne 
    in perspektivne. Pri paralelnih vzporedne daljice v prostoru obdržijo vzporednost tudi po 
    projekciji na ravnino. Pri perspektivnih pa se z oddaljenostjo od gledišča razdalje manjšajo.
    Pri perspektivnih projekcijah lahko ravne črte ostanejo ravne in se razdalje manjšajo samo v 
    eno smer, dve ali pa tri, lahko pa ravne črte postanejo krive.
    \cite{3D-projection}

\section{Računanje projekcije}
    Vidno polje \\
    \begin{tikzpicture}
        \coordinate (origin) at (0, 0);
        \node[anchor=north east] at (origin) {Mesto opazovalca};
        \coordinate (direction) at (10, 0);
        \coordinate (upfin) at (10, 5);
        \coordinate (botfin) at (10, -5);
        \draw[-stealth] (origin) -- (direction) node[anchor=west] {Smer pogleda};
        \draw[-stealth] (origin) -- (upfin);
        \draw[-stealth] (origin) -- (botfin);
        \pic [draw, angle eccentricity=1.5, angle radius=5cm] {angle = direction--origin--upfin};
        \node at (2,0.5) {$\varphi$};
    \end{tikzpicture}

% biber projekcija-teorija
\printbibliography

\end{document}