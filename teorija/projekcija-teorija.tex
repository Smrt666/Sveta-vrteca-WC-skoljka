\documentclass[a4paper, 12px]{article}
\usepackage{amsmath}
\usepackage{amssymb}
\usepackage{amsthm}
\usepackage{amsfonts}
\usepackage{gensymb}

\usepackage{biblatex}
\addbibresource{viri-projekcija.bib}

\usepackage{witharrows}

\usepackage{tikz}
\usetikzlibrary{automata, positioning, angles, quotes, calc, patterns}

\usepackage[utf8]{inputenc}
\usepackage[slovene]{babel}

\title{Del teorije pri projektnem delu v drugem letniku}

\begin{document}
\maketitle
\tableofcontents

\section{Projekcija}
\subsection{Kaj je 3D projekcija}
    3D projekcija oziroma grafična projekcija je tehnika, ki se uporablja za prikaz 3 dimenzionalnih objektov na 
    2 dimenzionalno ravnino. Obstaja več različnih vrst projekcij. V grobem se delijo na paralelne 
    in perspektivne. Pri paralelnih vzporedne daljice v prostoru obdržijo vzporednost tudi po 
    projekciji na ravnino. Pri perspektivnih pa se z oddaljenostjo od gledišča razdalje manjšajo.
    Pri perspektivnih projekcijah lahko ravne črte ostanejo ravne in se razdalje manjšajo samo v 
    eno smer, dve ali pa tri, lahko pa tudi ravne črte, ki ne ležijo na oseh, postanejo krive.
    \cite{3D-projection}

\subsection{Računanje projekcije}
    Skica \\
    \begin{tikzpicture}
        \coordinate (origin) at (0, 0);
        \node[anchor=north east] at (origin) {$O\left(0, 0\right)$};
        \coordinate (direction) at (10, 0);
        \coordinate (upfin) at (10, 5);
        \coordinate (botfin) at (10, -5);
        \draw[-] (2, 1) -- (2, -1) node[anchor=south west] {Zaslon}; % Tukaj je zaslon
        \draw[-stealth] (origin) -- (direction) node[anchor=west] {Smer pogleda ($y$)};
        \draw[-stealth] (0, -5) -- (0, 5) node[anchor=west] {Smer $x$};
        \draw[-stealth] (origin) -- (upfin);
        \draw[-stealth] (origin) -- (botfin);
        \pic [draw, angle eccentricity=1.5, angle radius=5cm] {angle = direction--origin--upfin};
        \node at (4,1) {$\varphi$};
        \draw (7,3.5) node[circle,fill,inner sep=1pt,label=above:\text{$T_0\left(x_0, y_0\right)$}](a){};
        \draw (7,0) node[circle,fill,inner sep=1pt,label=above:\text{$T_1\left(x_1, y_1\right)$}](a){};
        \draw (7,2) node[circle,fill,inner sep=1pt,label=above:\text{$T_2\left(x_2, y_2\right)$}](a){};
        \draw (8,2.2857) node[circle,fill,inner sep=1pt,label=right:\text{$T_3\left(x_3, y_3\right)$}](a){};
        \draw (8,4) node[circle,fill,inner sep=1pt,label=right:\text{$T_4\left(x_4, y_4\right)$}](a){};

        \draw (2,1) node[circle,fill,inner sep=1pt,label=right:\text{$T'_0$}](a){};
        \draw (2,0) node[circle,fill,inner sep=1pt,label=right:\text{$T'_1$}](a){};
        \draw (2,0.5714) node[circle,fill,inner sep=1pt,label=right:\text{$T'_{2,3}$}](a){};
        \draw[-stealth, dotted] (origin) -- (10, 2.857);
    \end{tikzpicture}\\
    Da si lažje predstavljamo kako točke projicirati, si problem narišemo v ravnini.
    Določimo mesto opazovalca, najlažje kar v točki $O\left(0, 0\right)$. 
    Vedeti moramo tudi širino zaslona $w$ in njegovo višino $h$. Razmerje med stranicama zaslona naj bo $a = \frac{h}{w}$.
    Velikost vidnega polja je v primeru na sliki $2\varphi$. Edini pomen tega kota je povečevanje in manjšanje slike.
    Mirno ga lahko izpustimo, če pa želimo s kotom določati velikost predmetov na projekciji, ga spreminjamo od malo več
    kot $0\degree$ za največjo povečavo, do manj kot $180\degree$ za najmanjšo povečavo.\\

    Vse točke, ki so na isti premici, ki gre skozi točko $O$, se preslikajo na isto mesto na zaslonu.
    (Seveda morajo biti te točke pred opazovalcem, torej imeti pozitivno $y$ koordinato.)
    Zato vsem točkam koordinato $x$ (in $z$) delimo z največjo vidno, pri tisti oddaljenosti od zaslona, kar je
    določeno z velikostjo vidnega polja.
    (Primer na sliki ima 3 točke, vse za $y_0 = y_1 = y_2$ oddaljene od zaslona. $T_0$ ima največjo še vidno $x$
    koordinato, pri tej oddaljenosti od zaslona. Če bi bila večja, bi bila na sliki višje, kar je že izven vidnega polja.
    Ko delimo $x_0$ z $x_0$, dobimo $1$, ki je največja vrednost, ki jo lahko na takšen način dobimo, drugače točka ni vidna.
    Točke na $y$ osi imajo $x$ (in $z$) koordinato $0$, $0$ deljeno z največjim $x$ pri tisti oddaljenosti od $O$, 
    bo pa vedno nič. Edina težava je, če je ta točka pri $y = 0$, ampak takšnih primerov ne rišemo.
    Točki $T_2$ in $T_3$ ležita na isti premici skozi $O$, zato se morata projicirati na isto mesto na zaslonu.
    Iz podobnih trikotnikov sledi, da je razmerje $x_2 : y_2 = x_3 : y_3$ in $x_0 : y_0 = x_4 : y_4$,
    ker sta pa $y_0 = y_2$ in $y_3 = y_4$, sledi $x_3 : x_2 = x_4 : x_0$, kar pa pomeni $x_4 : x_3 = x_0 : x_2$.
    Skupaj s tem, da se ohranijo razmerja razdalj na premicah, vzporednih z $x$ osjo, pa to točno to, kar potrebujemo.)\\

    V splošnem to dosežemo tako, da koordinati $x$ (in $z$) delimo z $R\tan{\varphi}$. R je tukaj oddaljenost točke od zaslona,
    torej je v primeru na sliki $R = y$, v več dimenzijah bi bilo pa $R = \sqrt{y^2 + w^2 + c_4^2 + c_5^2 + \dots + c_{n-1}^2}$,
    kjer smo koordinate po vrsti označili z $x, y, z, w, c_4, c_5, \dots, c_{n-1}$, $n$ je pa število dimenzij.
    $R\tan{\varphi}$ je največja vidna koordinata $x$ (in $z$) pri dani oddaljenosti od zaslona $R$.
    S tem, ko smo vsako točko pomnožili z $s = \frac{1}{\tan{\varphi}}$ in delili z $R$, ima vsaka nova točka koordinati $x$ in $z$ med $-1$ in $1$. \\

    Preostane nam samo še postavljanje vidnih točk na zaslon $w \times h$. To dosežemo tako, da prvo koordinato pomnožimo
    z razmerjem stranic zaslona $a$, potem pa obema prištejemo $1$ in ju pomnožimo z velikostjo zaslona. Pri tem ne smemo
    pozabiti, da je to prištevanje $1$ ker je na zaslonih točka $\left(0, 0\right)$ v zgornjem levem kotu, $x$ koordinata
    se veča v desno, $y$ pa navzdol. Zaradi tega, moramo tudi drugače obrniti predmete, ki jih projiciramo, ali pa pomnožiti
    $z$ z $-1$. Torej se točka $T\left(x, y, z\right)$ projicira v 
    $T'\left(\left(\frac{h}{w} \cdot \frac{x}{y \cdot \tan{\varphi}} + 1\right) \cdot \frac{w}{2},
    \left(\frac{z}{y \cdot \tan{\varphi}} + 1\right) \cdot \frac{h}{2} \right)$, 
    kjer je $T'$ točka na zaslonu. Če preoblikujemo izraz, postane 
    $$T'\left(\left(a \cdot x \cdot s \cdot \frac{1}{R} + 1\right) \cdot \frac{w}{2},
    \left(z \cdot s \cdot \frac{1}{R} + 1\right) \cdot \frac{h}{2} \right)$$
    Koordinate točk $T'$ so v koordinatah zaslona, ne začetnega prostora.
    \cite{ProjectionVideo}

\section{Manipulacija prostora}
    V prejšnjem poglavju je opisan postopek projiciranja točk na ravnino $xz$.
    Ker pa se lahko smer in položaj opazovalca spreminjata, je potrebno temu primerno premikati
    točke po prostoru, da se doseže učinek premikanja opazovalca.
\subsection{Vektorji}
    Za predstavitev točk v prostoru se pogosto uporablja krajevne vektorje. Primer:
    $\vec{v}=\begin{bmatrix}
        x \\ y \\ z
    \end{bmatrix}$, če je točka del 3 dimenzionalnega prostora. Vektorje lahko seštevamo in odštevamo:
    $\begin{bmatrix}
        x_1 \\ y_1 \\ z_1
    \end{bmatrix} - \begin{bmatrix}
        x_2 \\ y_2 \\ z_2
    \end{bmatrix} = \begin{bmatrix}
        x_1 - x_2 \\ y_1 - y_2 \\ z_1 - z_2
    \end{bmatrix}$, lahko pa tudi množimo. \\

    Vektorje lahko množimo na več različnih načinov. Trije najbolj poznani načini so produkt s skalarjem, 
    skalarni produkt in vektorski produkt. Produkt vektorja s skalarjem vrne vektor enake velikosti:
    $$k \begin{bmatrix}
        x \\ y \\ z \\ \vdots
    \end{bmatrix} = \begin{bmatrix}
        k x \\ k y \\ k z \\ \vdots
    \end{bmatrix}$$
    Kadar imamo opravka z realnimi števili, operacija poveča ali zmanjša vektor za nek faktor. \\

    Vektorski produkt je možen samo v treh in sedmih dimenzijah. Za prostore ostalih dimenzij obstajajo
    druge operacije, ki imajo podobne lastnosti kot vektorski produkt.
    $$\begin{bmatrix}
        A_x \\ A_y \\ A_z
    \end{bmatrix} \times \begin{bmatrix}
        B_x \\ B_y \\ B_z
    \end{bmatrix} = \begin{bmatrix}
        A_y B_z - A_z B_y \\ A_z B_x - A_x B_z \\ A_x B_y - A_y B_x
    \end{bmatrix}$$
    V dveh dimenzijah se lahko pretvarjamo, da je $z$ koordinata enaka $0$. S tem prvi dve vrednosti
    produkta postaneta $0$, predznak zadnje pa nam pove orientacijo trikotnika, ki ga ta dva vektorja 
    določata in dvakratnik njegove ploščine. Predznak lahko uporabimo za določanje konveksnosti likov
    s podanimi oglišči ali pa če stranica 3D objekta gleda v našo smer.
\subsection{Matrike}
\subsection{Rotacije}
\subsection{Translacije}

% biber projekcija-teorija
\printbibliography

\end{document}