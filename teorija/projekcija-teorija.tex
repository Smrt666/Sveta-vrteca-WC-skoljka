\documentclass[a4paper, 12px]{article}
\usepackage{amsmath}
\usepackage{amssymb}
\usepackage{amsthm}
\usepackage{amsfonts}
\usepackage{gensymb}

\usepackage{biblatex}
\addbibresource{viri-projekcija.bib}

\usepackage{witharrows}

\usepackage{tikz}
\usetikzlibrary{automata, positioning, angles, quotes, calc, patterns}

\usepackage[utf8]{inputenc}
\usepackage[slovene]{babel}

\title{Del teorije pri projektnem delu v drugem letniku}

\begin{document}
\maketitle
\tableofcontents

\section{Projekcija}
\subsection{Kaj je 3D projekcija}
    3D projekcija oziroma grafična projekcija je tehnika, ki se uporablja za prikaz 3 dimenzionalnih objektov na 
    2 dimenzionalno ravnino. Obstaja več različnih vrst projekcij. V grobem se delijo na paralelne 
    in perspektivne. Pri paralelnih vzporedne daljice v prostoru obdržijo vzporednost tudi po 
    projekciji na ravnino. Pri perspektivnih pa se z oddaljenostjo od gledišča razdalje manjšajo.
    Pri perspektivnih projekcijah lahko ravne črte ostanejo ravne in se razdalje manjšajo samo v 
    eno smer, dve ali pa tri, lahko pa tudi ravne črte, ki ne ležijo na oseh, postanejo krive.
    \cite{3D-projection}

\subsection{Računanje projekcije}
    Skica \\
    \begin{tikzpicture}
        \coordinate (origin) at (0, 0);
        \node[anchor=north east] at (origin) {$O\left(0, 0\right)$};
        \coordinate (direction) at (10, 0);
        \coordinate (upfin) at (10, 5);
        \coordinate (botfin) at (10, -5);
        \draw[-] (2, 1) -- (2, -1) node[anchor=south west] {Zaslon}; % Tukaj je zaslon
        \draw[-stealth] (origin) -- (direction) node[anchor=west] {Smer pogleda ($y$)};
        \draw[-stealth] (0, -5) -- (0, 5) node[anchor=west] {Smer $x$};
        \draw[-stealth] (origin) -- (upfin);
        \draw[-stealth] (origin) -- (botfin);
        \pic [draw, angle eccentricity=1.5, angle radius=5cm] {angle = direction--origin--upfin};
        \node at (4,1) {$\varphi$};
        \draw (7,3.5) node[circle,fill,inner sep=1pt,label=above:\text{$T_0\left(x_0, y_0\right)$}](a){};
        \draw (7,0) node[circle,fill,inner sep=1pt,label=above:\text{$T_1\left(x_1, y_1\right)$}](a){};
        \draw (7,2) node[circle,fill,inner sep=1pt,label=above:\text{$T_2\left(x_2, y_2\right)$}](a){};
        \draw (8,2.2857) node[circle,fill,inner sep=1pt,label=right:\text{$T_3\left(x_3, y_3\right)$}](a){};
        \draw (8,4) node[circle,fill,inner sep=1pt,label=right:\text{$T_4\left(x_4, y_4\right)$}](a){};

        \draw (2,1) node[circle,fill,inner sep=1pt,label=right:\text{$T'_0$}](a){};
        \draw (2,0) node[circle,fill,inner sep=1pt,label=right:\text{$T'_1$}](a){};
        \draw (2,0.5714) node[circle,fill,inner sep=1pt,label=right:\text{$T'_{2,3}$}](a){};
        \draw[-stealth, dotted] (origin) -- (10, 2.857);
    \end{tikzpicture}\\
    Da si lažje predstavljamo kako točke projicirati, si problem narišemo v ravnini.
    Določimo mesto opazovalca, najlažje kar v točki $O\left(0, 0\right)$. 
    Vedeti moramo tudi širino zaslona $w$ in njegovo višino $h$. Razmerje med stranicama zaslona naj bo $a = \frac{h}{w}$.
    Velikost vidnega polja je v primeru na sliki $2\varphi$. Edini pomen tega kota je povečevanje in manjšanje slike.
    Mirno ga lahko izpustimo, če pa želimo s kotom določati velikost predmetov na projekciji, ga spreminjamo od malo več
    kot $0\degree$ za največjo povečavo, do manj kot $180\degree$ za najmanjšo povečavo.\\

    Vse točke, ki so na isti premici, ki gre skozi točko $O$, se preslikajo na isto mesto na zaslonu.
    (Seveda morajo biti te točke pred opazovalcem, torej imeti pozitivno $y$ koordinato.)
    Zato vsem točkam koordinato $x$ (in $z$) delimo z največjo vidno, pri tisti oddaljenosti od zaslona, kar je
    določeno z velikostjo vidnega polja.
    (Primer na sliki ima 3 točke, vse za $y_0 = y_1 = y_2$ oddaljene od zaslona. $T_0$ ima največjo še vidno $x$
    koordinato, pri tej oddaljenosti od zaslona. Če bi bila večja, bi bila na sliki višje, kar je že izven vidnega polja.
    Ko delimo $x_0$ z $x_0$, dobimo $1$, ki je največja vrednost, ki jo lahko na takšen način dobimo, drugače točka ni vidna.
    Točke na $y$ osi imajo $x$ (in $z$) koordinato $0$. Ko $0$ delimo z največjim $x$ pri tisti oddaljenosti od $O$, 
    vedno dobimo $0$. Edina težava je, če je ta točka pri $y = 0$, ampak takšnih primerov ne rišemo.
    Točki $T_2$ in $T_3$ ležita na isti premici skozi $O$, zato se morata projicirati na isto mesto na zaslonu.
    Iz podobnih trikotnikov sledi, da je razmerje $x_2 : y_2 = x_3 : y_3$ in $x_0 : y_0 = x_4 : y_4$,
    ker sta pa $y_0 = y_2$ in $y_3 = y_4$, sledi $x_3 : x_2 = x_4 : x_0$, kar pa pomeni $x_4 : x_3 = x_0 : x_2$.
    Skupaj s tem, da se ohranijo razmerja razdalj na premicah, vzporednih z $x$ osjo, je to točno to, kar potrebujemo.)\\

    V splošnem to dosežemo tako, da koordinati $x$ (in $z$) delimo z $R\tan{\varphi}$. R je tukaj oddaljenost točke od zaslona,
    torej je v primeru na sliki $R = y$, v več dimenzijah bi bilo pa $R = \sqrt{y^2 + w^2 + c_4^2 + c_5^2 + \dots + c_{n-1}^2}$,
    kjer smo koordinate po vrsti označili z $x, y, z, w, c_4, c_5, \dots, c_{n-1}$, $n$ je pa število dimenzij.
    $R\tan{\varphi}$ je največja vidna koordinata $x$ (in $z$) pri dani oddaljenosti od zaslona $R$.
    S tem, ko smo vsako točko pomnožili z $s = \frac{1}{\tan{\varphi}}$ in delili z $R$, ima vsaka nova točka koordinati $x$ in $z$ med $-1$ in $1$. \\

    Zdaj lahko govorimo o točkah na ravnini, s koordinatama $x$ in $y$, ki pa se ne ujemata s tistimi iz začetnega koordinatnega sistema.
    Preostane nam samo še postavljanje vidnih točk na zaslon $w \times h$. Recimo, da želimo imeti na zaslonu
    točke s koordinatama med $-1$ in $1$. Če je zaslon kvadratne oblike ($h = w$), s tem ni težav. V nasprotnem primeru,
    v katerem je zaslon pravokotnik in ne kvadrat, bi dobili sliko raztegnjeno po večji stranici zaslona. Zaradi tega problema je 
    lahko ena koordinata, recimo $y$, med $-1$ in $1$, $x$ koordinata pa med $-1 \cdot \frac{w}{h} = -\frac{1}{a}$ in $\frac{1}{a}$.
    Da bodo imele vse točke na zaslonu koordinati med $-1$ in $1$, ravnino raztegnemo po $x$ osi tako, da končna slika ne bo 
    raztegnjena. To dosežemo tako, da $x$ koordinato pomnožimo z razmerjem stranic zaslona $a$.\\

    Na zaslonih je ponavadi točka $\left(0, 0\right)$ v zgornjem levem kotu, $x$ koordinata se veča v desno, $y$ pa navzdol.
    Zato obema koordinatama točk prištejemo $1$, $y$ koordinati pa še pred tem zamenjamo predznak.
    Nato ju pomnožimo s polovico velikosti zaslona ($y \cdot \frac{1}{2} \cdot h$ in $x \cdot \frac{1}{2} \cdot w$), tako da celoten zaslon zapolnijo točke med $-1$ in $1$. \\

    Če povzamem, se točka $T\left(x, y, z\right)$ projicira v 
    $T'\left(\left(\frac{h}{w} \cdot \frac{x}{y \cdot \tan{\varphi}} + 1\right) \cdot \frac{w}{2},
    \left(\frac{z}{y \cdot \tan{\varphi}} + 1\right) \cdot \frac{h}{2} \right)$, 
    kjer je $T'$ točka na zaslonu. Če preoblikujemo izraz, postane 
    $$T'\left(\left(a \cdot x \cdot s \cdot \frac{1}{R} + 1\right) \cdot \frac{w}{2},
    \left(z \cdot s \cdot \frac{1}{R} + 1\right) \cdot \frac{h}{2} \right)$$
    Koordinate točk $T'$ so v koordinatah zaslona, ne začetnega prostora.
    \cite{ProjectionVideo}

\section{Manipulacija prostora}
    V prejšnjem poglavju je opisan postopek projiciranja točk na ravnino $xz$.
    Ker pa se lahko smer in položaj opazovalca spreminjata, je potrebno temu primerno premikati
    točke po prostoru, da se doseže učinek premikanja opazovalca.
\subsection{Vektorji}
    Za predstavitev točk v prostoru se pogosto uporablja krajevne vektorje. Primer:
    $\vec{v}=\begin{bmatrix}
        x \\ y \\ z
    \end{bmatrix}$, če je točka del 3 dimenzionalnega prostora. Vektorje lahko seštevamo in odštevamo:
    $\begin{bmatrix}
        x_1 \\ y_1 \\ z_1
    \end{bmatrix} - \begin{bmatrix}
        x_2 \\ y_2 \\ z_2
    \end{bmatrix} = \begin{bmatrix}
        x_1 - x_2 \\ y_1 - y_2 \\ z_1 - z_2
    \end{bmatrix}$, lahko pa tudi množimo. \\

    Vektorje lahko množimo na več različnih načinov. Trije najbolj poznani načini so produkt s skalarjem, 
    skalarni produkt in vektorski produkt. Produkt vektorja s skalarjem vrne vektor enake velikosti:
    $$k \begin{bmatrix}
        x \\ y \\ z \\ \vdots
    \end{bmatrix} = \begin{bmatrix}
        k x \\ k y \\ k z \\ \vdots
    \end{bmatrix}$$
    Kadar imamo opravka z realnimi števili, operacija poveča ali zmanjša vektor za nek faktor. \\

    Vektorski produkt je možen samo v treh in sedmih dimenzijah. Za prostore ostalih dimenzij obstajajo
    druge operacije, ki imajo podobne lastnosti kot vektorski produkt.
    $$\begin{bmatrix}
        A_x \\ A_y \\ A_z
    \end{bmatrix} \times \begin{bmatrix}
        B_x \\ B_y \\ B_z
    \end{bmatrix} = \begin{bmatrix}
        A_y B_z - A_z B_y \\ A_z B_x - A_x B_z \\ A_x B_y - A_y B_x
    \end{bmatrix}$$
    V dveh dimenzijah se lahko pretvarjamo, da je $z$ koordinata enaka $0$. S tem prvi dve vrednosti
    produkta postaneta $0$, predznak zadnje pa nam pove orientacijo trikotnika, ki ga ta dva vektorja 
    določata in dvakratnik njegove ploščine. S tem predznakom lahko ugotovimo ali stranica 3D objekta gleda v našo smer.
    \cite{CrossProdWiki}
\subsection{Matrike}
    Matrika je enostavno rečeno preglednica števil. Matrike lahko seštevamo, če so enakih velikosti:
    $$\begin{bmatrix}
        a_{1,1} & a_{1,2} & a_{1,3} \\
        a_{2,1} & a_{2,2} & a_{2,3} \\
        a_{3,1} & a_{3,2} & a_{3,3} \\
        a_{4,1} & a_{4,2} & a_{4,3} \\
    \end{bmatrix} + \begin{bmatrix}
        b_{1,1} & b_{1,2} & b_{1,3} \\
        b_{2,1} & b_{2,2} & b_{2,3} \\
        b_{3,1} & b_{3,2} & b_{3,3} \\
        b_{4,1} & b_{4,2} & b_{4,3} \\
    \end{bmatrix} = \begin{bmatrix}
        a_{1,1}+b_{1,1} & a_{1,2}+b_{1,2} & a_{1,3}+b_{1,3} \\
        a_{2,1}+b_{2,1} & a_{2,2}+b_{2,2} & a_{2,3}+b_{2,3} \\
        a_{3,1}+b_{3,1} & a_{3,2}+b_{3,2} & a_{3,3}+b_{3,3} \\
        a_{4,1}+b_{4,1} & a_{4,2}+b_{4,2} & a_{4,3}+b_{4,3} \\
    \end{bmatrix}$$
    Prav tako jih lahko množimo s skalarjem:
    $$k \begin{bmatrix}
        a_{1,1} & a_{1,2} & a_{1,3} \\
        a_{2,1} & a_{2,2} & a_{2,3} \\
        a_{3,1} & a_{3,2} & a_{3,3} \\
        a_{4,1} & a_{4,2} & a_{4,3}
    \end{bmatrix} = \begin{bmatrix}
        k \cdot a_{1,1} & k \cdot a_{1,2} & k \cdot a_{1,3} \\
        k \cdot a_{2,1} & k \cdot a_{2,2} & k \cdot a_{2,3} \\
        k \cdot a_{3,1} & k \cdot a_{3,2} & k \cdot a_{3,3} \\
        k \cdot a_{4,1} & k \cdot a_{4,2} & k \cdot a_{4,3}
    \end{bmatrix}$$
    Matriko z višino $a$ in širino $b$ lahko pomnožimo z matriko z višino $b$ in širino $c$. Rezultat je matrika
    z višino $a$ in širino $c$. Pri tem je vrednost v vrstici $i$ v stolpcu $j$ enaka skalarnemu produktu
    vrstice $i$ prve matrike s stolpcem $j$ druge. (Obstajajo tudi drugi tipi množenja, ampak za delo z vektorji bo ta dovolj.)
    \cite{MatrixWiki}
    $$\begin{bmatrix}
        a_{1,1} & a_{1,2} \\ a_{2,1} & a_{2,2}
    \end{bmatrix} \begin{bmatrix}
        b_{1,1} & b_{1,2} \\ b_{2,1} & b_{2,2}
    \end{bmatrix} = \begin{bmatrix}
        a_{1,1} \cdot b_{1,1} + a_{1,2} \cdot b_{2,1} & a_{1,1} \cdot b_{1,2} + a_{1,2} \cdot b_{2,2} \\
        a_{2,1} \cdot b_{1,1} + a_{2,2} \cdot b_{2,1} & a_{2,1} \cdot b_{1,2} + a_{2,2} \cdot b_{2,2}
    \end{bmatrix}$$
    Vektor si lahko zapišemo kot matriko s širino 1. Matriko lahko potem pomnožimo z vektorjem:
    $$\begin{bmatrix}
        a_{1,1} & a_{1,2} & a_{1,3} \\
        a_{2,1} & a_{2,2} & a_{2,3} \\
        a_{3,1} & a_{3,2} & a_{3,3}
    \end{bmatrix} \begin{bmatrix}
        x \\ y \\ z
    \end{bmatrix} = \begin{bmatrix}
        a_{1,1} \cdot x + a_{1,2} \cdot y + a_{1,3} \cdot z\\
        a_{2,1} \cdot x + a_{2,2} \cdot y + a_{2,3} \cdot z\\
        a_{3,1} \cdot x + a_{3,2} \cdot y + a_{3,3} \cdot z
    \end{bmatrix}$$
    Množenje matrik ni komutativno, torej $AB = BA$ ne velja vedno, pri njihovi uporabi se je potrebno tega zavedati. 
    Je pa asociativno, torej $ABC = (AB)C = A(BC)$.\\

    Na matrikah je definiranih še veliko drugih operacij, deljenja z matriko pa med njimi ni. Namesto tega
    lahko kvadratni matriki (z ne ničelno determinanto) izračunamo inverz, ki se zapiše kot $A^{-1}$. Velja,
    da množenje matrike z inverzom vrne identiteto. Identiteta je matrika, ki ima po diagonali od levo zgoraj do desno
    spodaj same enke, ostala polja pa imajo za vrednost $0$. $AI = A$ velja, če je I identiteta, $A$ pa poljubna matrika.
    \cite{MatrixWiki}
    
\subsection{Rotacije}
    \begin{tikzpicture}
        \coordinate (origin) at (0,0);
        \coordinate (very_left) at (-4, 0);
        \coordinate (very_right) at (5, 0);
        \coordinate (very_top) at (0, 5);
        \coordinate (very_bottom) at (0, -3.5);
        \node[anchor=north east] at (origin) {$\left(0, 0\right)$};
        \draw[-stealth] (very_left) -- (very_right) node[anchor=west] {$x$};
        \draw[-stealth] (very_bottom) -- (very_top) node[anchor=west] {$y$};
        \coordinate (unit_i) at (3, 0);
        \coordinate (unit_j) at (0, 3);
        \draw (unit_i) node[circle, fill,inner sep=1pt, label=below: \text{$i$}](a){};
        \draw (unit_j) node[circle, fill,inner sep=1pt, label=right: \text{$j$}](a){};

        \draw (2.598, 1.5) node[circle, fill,inner sep=1pt, label=right: \text{$i'$}](a){};
        \coordinate (angle) at (5.196, 3);
        \pic [draw, angle eccentricity=1.5, angle radius=2cm] {angle = unit_i--origin--angle};
        \node at (1.2,0.4) {$\varphi$};
        \draw (origin) -- (angle);
        \draw (2.598,0) node[circle, fill,inner sep=1pt, label=below: \text{$i'_x$}](a){};
        \draw (0,1.5) node[circle, fill,inner sep=1pt, label=left: \text{$i'_y$}](a){};
        \draw[dotted] (2.598,0) -- (2.598, 1.5);
        \draw[dotted] (0,1.5) -- (2.598, 1.5);

        \draw (-1.5, 2.598) node[circle, fill,inner sep=1pt, label=left: \text{$j'$}](a){};
        \coordinate (angle) at (-3, 5.196);
        \pic [draw, angle eccentricity=1.5, angle radius=2cm] {angle = unit_j--origin--angle};
        \node at (-0.3,1) {$\varphi$};
        \draw (origin) -- (angle);
        \draw (0, 2.598) node[circle, fill,inner sep=1pt, label=right: \text{$j'_y$}](a){};
        \draw (-1.5,0) node[circle, fill,inner sep=1pt, label=below: \text{$j'_x$}](a){};
        \draw[dotted] (0, 2.598) -- (-1.5, 2.598);
        \draw[dotted] (-1.5, 0) -- (-1.5, 2.598);

        \node[circle, draw, scale=18.1, dotted] (c) at (0,0){};    
    \end{tikzpicture} \\
    Pri rotaciji za kot $\varphi$ se enota na osi $x$ preslika v točko $(\cos\varphi, \sin\varphi)$,
    enota na osi $y$ pa se preslika v točko $(-\sin\varphi, \cos\varphi)$. Kot rotacijsko matriko lahko to zapišemo:
    $$\begin{bmatrix}
        \cos\varphi & - \sin\varphi \\
        \sin\varphi & \cos\varphi
    \end{bmatrix}$$
    Vektor zavrtimo za kot $\varphi$ tako, da matriko pomnožimo z njim.
    $$\begin{bmatrix}
        \cos\varphi & - \sin\varphi \\
        \sin\varphi & \cos\varphi
    \end{bmatrix} \begin{bmatrix}
        x \\ y
    \end{bmatrix} = \begin{bmatrix}
        x \cos\varphi - y \sin\varphi \\
        x \sin\varphi + y \cos\varphi
    \end{bmatrix}$$
    Rotacijo v nasprotno smer dobimo, če vstavimo negativen $\varphi$, kar spremeni predznak pri obeh $\sin$. 
    \cite{TransformationWiki}\\
    
    Če želimo izvajati rotacije v višjih dimenzijah, potrebujemo večje matrike. V treh dimenzijah pravimo,
    da zavrtimo nekaj okrog neke osi. V štirih dimenzijah vrtimo okrog ravnine, v petih okrog prostora, v šestih
    okrog 4D hiper prostora\dots\\

    V petih dimenzijah bi rotacijska matrika, ki vrti v smeri od $y$ osi proti $w$ osi izgledala takole 
    (osi si sledijo $x$, $y$, $z$, $w$, $k$):
    $$\begin{bmatrix}
        1 & 0 & 0 & 0 & 0 \\
        0 & \cos\varphi & 0 & -\sin\varphi & 0 \\
        0 & 0 & 1 & 0 & 0 \\
        0 & \sin\varphi & 0 & \cos\varphi & 0 \\
        0 & 0 & 0 & 0 & 1
    \end{bmatrix}$$\\

    Zdaj je na vrsti vprašanje, zakaj sploh uporabljati matrike, če lahko samo vzamemo dve koordinati, $\sin\varphi$ in $\cos\varphi$
    ter zapišemo novo točko. Odgovor ni samo lepši zapis, ampak tudi manjše število operacij, potrebnih,
    da vrtimo večje število točk (vektorjev). V prostoru z $N$ dimenzijami lahko vrtimo od katerekoli osi
    proti katerikoli drugi. To je $\frac{N(N-1)}{2}$ različnih rotacij, če štejemo rotacije v nasprotne smeri za enake.
    Vsakič lahko točke zavrtimo na $\frac{N(N-1)}{2}$ načinov, kar bi pomenilo prav toliko dela. Če zapišemo vseh
    $\frac{N(N-1)}{2}$ rotacijskih matrik za podan kot, bi dobili za rotacijo takšen izraz:
    $R_1 R_2 R_3 \dots R_{\frac{N(N-1)}{2}} V$. $V$ je v tem primeru vektor, ki ga obračamo.
    Kot pa vemo, je množenje matrik asociativno, torej lahko vse matrike zmnožimo vnaprej, da dobimo eno rotacijsko matriko,
    ki vsebuje vse rotacije. Potem lahko tisto eno matriko pomnožimo z vsakim vektorjem, kar nam prihrani veliko dela.\\

    Ker množenje matrik ni komutativno, moramo paziti, v kakšen vrstni red jih postavimo. Tiste najbolj desno
    bodo pomenile rotacije, ki se zvedejo najprej, tiste bolj levo, pa kasnejše.\cite{3D-projection}

\subsection{Translacije}
    Translacija ali premik je nekaj, kar lahko dosežemo s seštevanjem in odštevanjem vektorjev.
    Žal pa potem ne moremo zapisati ene matrike, ki bi jo pomnožili s poljubnim vektorjem.
    Prav tako, če imamo nek $N$-dimenzionalen vektor, matrika, ki bi jo pomnožili z njim, da bi ga prestavili,
    ne obstaja. Da se izognemo tej težavi, premaknemo vektor v eno dimenzijo višje ter mu nastavimo novo koordinato na $1$.
    Potem lahko translacijsko matriko zapišemo kot:
    $$\begin{bmatrix}
        1 & 0 & 0 & \cdots & \Delta x \\
        0 & 1 & 0 & \cdots & \Delta y \\
        0 & 0 & 1 & \cdots & \Delta z \\
        \vdots & \vdots & \vdots & \ddots & \vdots \\
        0 & 0 & 0 & \cdots & 1
    \end{bmatrix}$$
    Da bomo translacijsko lahko pomnožili z rotacijsko, moramo rotacijskim dodati na koncu en stolpec in eno vrstico,
    oba zapolnjena z ničlami, le skrajno desno spodnja vrednost mora biti $1$. Ko rišemo, rišemo samo vektorje z zadnjo koordinato 1.\\

    Ali bomo dali translacijsko matriko pred rotacijsko ali za njo, je odvisno od tega, ali želimo najprej izvesti
    rotacije in nato premik, ali obratno. Možno je tudi, da potrebujemo oboje, saj morda želimo vrteti okrog točke, ki
    ni koordinatno izhodišče.\\

    Število operacij, da izvedemo premike, bo v primeru da jih izvajamo samo enkrat, približno enako, kot če bi se tega 
    lotili z odštevanjem. Prednost tega pristopa se pokaže, če želimo večkrat prestaviti vektorje.\cite{TransformationWiki}

\subsection{Zrcaljenje}
    Za zrcaljenje skozi koordinatno izhodišče lahko samo vse koordinate pomnožimo z $-1$. Temu primerna
    matrika je identiteta pomnožena z $-1$:
    $$-1 \begin{bmatrix}
        1 & 0 & 0 & 0 \\
        0 & 1 & 0 & 0 \\
        0 & 0 & 1 & 0 \\
        0 & 0 & 0 & 1
    \end{bmatrix} = \begin{bmatrix}
        -1 & 0 & 0 & 0 \\
        0 & -1 & 0 & 0 \\
        0 & 0 & -1 & 0 \\
        0 & 0 & 0 & -1
    \end{bmatrix}$$
    Paziti moramo, če opravljamo translacije z matrikami, ker v tem primeru zadnja koordinata ne sme biti pomnožena
    z $-1$. V tem primeru moramo napisati matriko:
    $$\begin{bmatrix}
        -1 & 0 & 0 & 0 \\
        0 & -1 & 0 & 0 \\
        0 & 0 & -1 & 0 \\
        0 & 0 & 0 & 1
    \end{bmatrix}$$
    Če želimo zavrteti vektor okrog premice, ga je najlažje zavrteti tako, da premica čez katero zrcalimo,
    sovpada z eno od osi. Potem vse koordinate vektorja, razen tiste, ki jo predstavlja omenjena os, pomnožimo z $-1$.
    Na koncu še zavrtimo vektor nazaj. Podobno lahko zrcalimo preko ravnine, samo da v tem primeru pri množenju z $-1$
    izpustimo dve koordinati namesto ene.

% biber projekcija-teorija
\printbibliography

\end{document}